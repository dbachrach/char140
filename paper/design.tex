\section{\hoot Design \& Security Analysis}
\label{sec:design-sec}

In this section, we describe the \hoot protocol and analyze its security.

\subsection{Design}
\label{sec:design}

We now describe \hoot in detail. After giving a brief overview of the
\hoot  protocol, we describe how hashtags are generated to provide
collisions with other groups and how the message header and body are
constructed to enable efficient searching.
%We also analyze the security of messages.

\paragraphX{Protocol overview.} A complete \hoot message consists of a 
header and a message body. The header contains a group identifier
(a Twitter-style hashtag), an encryption key and a MAC key, both
encrypted with a session key, and finally a MAC
over the ciphertext of the message (see
Figure~\ref{fig:hoot-structure}). As in
Twitter, messages do not name their recipients. Anyone who knows the
secret hashtag associated with a \hoot can decrypt and read the message
as well as validate its integrity.
We also need an efficient discovery mechanism.
Rather than attempting to treat every message
posted to Twitter as a potential group message, and thus being
required to fetch and attempt decryption of every single message,
the \hoot protocol places an identifier into every \hoot message
as a hashtag so a fellow group member can simply search for the
identifier to see all potential messages. With a constant group
identifier, readers can also publicly follow that identifier like any
other hashtag on Twitter.

\paragraphX{Group identifiers.} To create a hashtag for use as the group
identifier, \hoot derives a short bitstring from the secret hashtag. We
must do this in such a way as to give an attacker no information about
the shared secret itself. A cryptographic hash function serves this
purpose well.
%Hash functions provide a great way to get a
%set of bits from a shared secret without divulging much information
%about the original shared secret. 
We call the secret hashtag a \textit{plain tag}, which is comparable to
a normal Twitter hashtag, though it should have enough entropy to
prevent the adversary from guessing it. The result of hashing the plain
tag with a given hash function \textit{H} (such as SHA-1) is referred to
as the \textit{long tag}, i.e.:
%
$\id{LongTag} \leftarrow H\left(\id{PlainTag}\right)$.

The \hoot protocol could simply use the long tag as an identifier, but
this choice leads to several problems. First, to achieve our design goal
of keeping identifiers short and to fit within Twitter's 140 character
limit, it is less than ideal to use the full output of a hash function
(e.g. 160 bits for SHA-1 or 256 bits for SHA-256). Secondly, strong hash
cryptographic functions produce virtually no collisions for reasonable
numbers of groups. As described in Section~\ref{sec:goals}, we propose
that different groups' identifiers collide with each other for recipient
anonymity and plausible deniability.

To generate a collision, we need to shorten the long tag, generating a
\textit{short tag} of $k$ bits. The short tag will, by design, induce
collisions between unrelated plain tags. The shorter the short tag, the
higher the collision rate will be and the less sure an observer can be
as to what topic a \hoot reader is actually following. With this greater
anonymity comes more computational work: since more group messages will
now belong to the same identifier, a follower must download and decrypt
more messages to find the desired ones.
%Depending on the required degree
%of subscriber anonymity, more collisions might be worth the
%computational overhead.
%Effectively, higher collision rates imply better anonymity but require
%downloading larger amounts of other groups' messages as cover
%traffic. 

Given a consistent system-wide short tag length, a group can choose a
tag that will collide with a popular tag, allowing for a predictably
high amount of cover traffic as well as providing a cover story for
followers of that tag.
% For example, if an Egyptian protest group wanted
% to find a collision with Justin Bieber's movie, we perform the following
% calculations:
%
\begin{figure}
\begin{codebox}
\Procname{$\proc{Find-Tag}(\id{prefix}, \id{target}, N, k):$}
\zi \For $i \gets [0,N)$, in random order
\zi \Do
\zi $\id{PlainTag} \gets  \id{prefix}.\id{suffix}$
\zi $\id{ShortTag} \gets H(\id{PlainTag}).\func{bits}(0 \ldots k-1)$
\zi \If $\id{ShortTag} = H(\id{target}).\func{bits}(0 \ldots k-1)$
\zi \Then $\func{return} (\id{PlainTag} , \id{ShortTag})$
\zi \End
\End
\end{codebox}
\caption{Pseudocode for tag collision searching.\label{fig:find-tag}}
\end{figure}

% \begin{align*}
% & \mathit{prefix} \leftarrow  ``\mathrm{\#Tahrir}'' \\
% & \mathit{target} \leftarrow ``\mathrm{\#nsn3d}'' \\
% & \forall_{i \in [0, N)}  \mathit{suffix} \leftarrow i \\
% & \mathit{PlainTag} \leftarrow  \mathit{prefix}.\mathit{suffix} \\
% & \mathit{ShortTag} \leftarrow H({\mathit PlainTag}).\mathrm{bits}(0 \ldots k-1) \\
% & \mathbf{if}\ {\mathit ShortTag} = H(\mathit{target}).\mathrm{bits}(0 \ldots k-1)\\
% & \mathbf{then}\ \mathrm{emit} (\mathit{PlainTag} , {\mathit{ShortTag}}) 
% \end{align*}
%
This algorithm searches for a tag collision, where the \id{PlainTag} suffix is
a number between 0 and $N$, and the \id{ShortTag} is $k$ bits long.
What should be reasonable values for $N$ and $k$?

$k$ determines the length of the \id{ShortTag}. As discussed above, the value for $k$
trades off anonymity versus search overhead for a receiver. $k$ will likely need to be
a constant shared widely across the space of \hoot users.

$N$ is bounded by how large a \id{PlainTag} string can be reasonably
passed among potential \hoot participants. If the communication of the
\id{PlainTag} must happen by word of mouth, $N$ will be bounded,
perhaps, by the number of digits that can be memorized by most humans
(so, if humans can remember around seven decimal
digits~\cite{miller56}, then $N$ would be $10^7$). Equivalently, we
could search over some other memorizable namespace with suitably high
entropy, like a short string of characters found on a keyboard.
Regardless, the group creator would use a \proc{Find-Tag} procedure (see
Figure~\ref{fig:find-tag}) to search over all possible suffixes to
identify hash collisions. Note that the search should be done
randomly, rather than in-order, to increase the attacker's difficulty
in conducting brute force attacks. Also not that process is only
necessary once, when a tag is first created. 

To further increase the entropy of the plain tag, we can imagine a
number of
options that would still be amenable to human memorization. For
example, the short tag's prefix could be
chosen randomly from a large dictionary.
If we were willing to relax our desire to have human-memorizable
plain tags, then the whole plain tag could be selected at random.
Certainly, this yields excellent resistance to brute force searching
attacks,
but it also creates additional complexity
for organizers wishing to prevent leaks, since these plain tags will need
to be written down~\cite{written-passwords}.

\paragraphX{Message header and body.}  
%
In addition to the \id{ShortTag}, the header contains a pair of
session keys
for message body encryption ($k_{\func{enc}}$) and integrity
verification ($k_{\func{mac}}$).

For every \hoot message, these session keys are randomly generated. Since we intend
to use efficient symmetric key ciphers and hash-based message authentication
functions. 
%, our random session keys should be selected to match the key
% lengths of our ciphers (e.g., 128 bits for AES). 
The session keys
are then encrypted with a {\em tag key} derived from the long tag, using
different bits than the $k$ bits used when deriving the
short tag. Given a long tag of 160 bits, if we assume half of those bits
are used in the short tag, the remaining 80 bits give
us $2^{80}$ possible keys that an attacker must potentially brute
force, which is certainly greater than the entropy in the
plaintext tag. (In Section~\ref{sec:experiments}, we flesh this out
in more detail.) Of course, if we ever reached a point where the
encryption and MAC session keys required more
bits than we can get from carving up the long tag, we could always
use the long tag to initialize a suitably strong pseudo-random number
generator, getting us all the derived bits we might ever want.

%In summary, the \hoot client takes plaintext message $M$ and a plain tag
%({\em PlainTag}) and constructs the \hoot messages as follows:
%Putting it all together, starting from a plaintext message $M$ with {\em
%  PlainTag} within it \hl{this sentence doesn't make sense to me}, a
%complete \hoot message will be as follows:
%

\begin{figure}
\begin{eqnarray*}
M & \leftarrow & \mathrm{plaintext\ message,\ including} \id{PlainTag}
\\
\id{LongTag} & \leftarrow & H(\id{PlainTag}) \\
\id{ShortTag} & \leftarrow & \id{LongTag}.\func{bits}(0 \ldots k-1) \\
k_{\func{tag}} & \leftarrow & \id{LongTag}.\func{bits}(k \ldots) \\
k_{\func{enc}}, k_{\func{mac}} & \leftarrow & \id{random bits} \\
C & \leftarrow & E_{k_{\func{enc}}}(M) \\
\id{HooT}  & \leftarrow &  \left(\id{ShortTag}, E_{k_{\func{tag}}}\left(k_{\func{enc}}, k_{\func{mac}}\right), \func{MAC}_{k_{\func{mac}}}(C), C\right)
\end{eqnarray*}
\caption{Structure of a \hoot message.\label{fig:hoot-structure}}
\end{figure}
%

So far, we have specified a \hoot structure with exactly one plain tag.
This technique can easily be generalized to support multiple plain tags. For
each one, a separate long tag can be generated, resulting in multiple
tag keys ($k_{\func{tag}}$), each of which is used to encrypt the same
session keys. The final \hoot would have multiple short tags and multiple
encryptions of the session keys, but only one ciphertext message
payload.

\begin{table*}
\caption{This table shows how several tweets might be converted to
  \hoot messages, showing the long tag, the short tag, and the final
  \hoot. The fourth message in this list demonstrates how an
  organization could take advantage of the \hoot system to collide its
  messages with those of an unrelated tag used for non-controversial
  messages.
\label{tab:process}
}
\begin{center}
    \begin{tabular}{ l  l  l  l  l }
	 & Tweet & Long Tag & Short Tag & Hoot \\ \hline
	1 & Its all bout the {\bf \#bieber} 100\%Belieber                                 & {\tt 9txrq71tfn8} &  {\tt 9tx} & {\tt \#9tx Xrtfn}... \\
	2 & Don't be a drag; just be a queen whether you're broke or {\bf \#CharlieSheen} & {\tt 7prQnd121f2} & {\tt 7pr} & {\tt \#7pr n771r}... \\
	3 & {\bf \#free-egypt} We'll meet at the usual, 11pm.                             & {\tt 2p7rtfx9pa1} & {\tt 2p7} & {\tt \#2p7 pp76a}... \\
	4 & {\bf \#free-egypt-9rqt} We'll meet at the usual, 11pm.                        & {\tt 9tx79srpLtt} &  {\tt 9tx}  & {\tt \#9tx 18yyQ}... \\
    \end{tabular}
\end{center}
\end{table*}

For illustration, Table~\ref{tab:process} shows how a few plain-text
tweets might be converted into their corresponding \hoot messages. The
first two messages are regular tweets from popular hashtags: \#bieber
and \#CharlieSheen. The third is a \hoot where receiver anonymity is
critical, but it's short tag, {\tt \#2p7}, does not collide with
anything else, and thus subscribers to {\tt \#2p7} might risk
discovery. The fourth message shows how the same group might alter
their plain tag so they can deliberate collide with {\tt \#bierber},
which maps to the same short tag ({\tt \#9tx}).


\subsection{Security Analysis}
\label{sec:security}

Based on the threat model defined in Section~\ref{sec:threat} and the
system design goals described in Section~\ref{sec:goals}, we now analyze
the security of the proposed \hoot protocol.

\paragraphX{Message security.}  
%
We begin with a brief analysis of the security of the message protocol
itself.
%We
%analyze the broader security of the \hoot scheme later in
%Section~\ref{sec:something-else}.

First, note that the session keys are generated randomly and
independently for each message, so that two identical plaintext messages
will have different ciphetexts. If the encryption scheme in use requires
an initialization vector (e.g., CBC mode), this could be included in the
message header. For other encryption schemes, such as counter mode, no
IV is necessary and the randomness of the key will ensure the
non-determinism of the ciphertext.

Message integrity is validated with a symmetric-key message
authentication code such as HMAC. Because the MAC is computed over the
ciphertext, and the MAC key is also generated at random, the MAC leaks
absolutely no information about the plaintext. The MAC verification
process also serves the purpose of identifying whether a prospective
message matches the plain tag in question (for which multiple other
plain tags will collide in the short tags), or whether a message is
irrelevant to the user's plain tag search query and should be dropped.

Replay attacks can be defeated by treating the session keys
($k_{\func{enc}}$, $k_{\func{mac}}$) as nonces. It's highly unlikely
that two different messages will share the same session keys.

%Consequently, the basic \hoot message encryption scheme appears to be
%sound, with the only obvious weakness being the selection of the plain
%tag.

\paragraphX{Dictionary Attacks and Brute Force.}  
%
Provided an attacker knows a targeted group's prefix and the alphabet
out of which they generate the suffix, our scheme is vulnerable to brute
force.

If we consider the adversary to be a government for example, it would be fair to assume they have
thousands of machines at their disposal for such an attack. It becomes
important, then for a user to pick a plain tag outside reasonable brute
forcing bounds.


\paragraphX{Traffic Analysis.}  
%

\hlfxnote{Sudden increases in a group's membership.}
\hlfxnote{Chris - do you mean for the above to be under Traffic Analysis or in a separate section?}
\hlfixme{Need to discuss all the features and attacks described in
  Section 2, at least briefly.}

