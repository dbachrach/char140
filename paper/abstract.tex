\begin{abstract}

%This paper presents the design and engineering of a system called
%HooT. 
Microblogging services such as Twitter are an increasingly important way
to communicate, both for individuals and for groups through the use of
hashtags that denote topics of conversation. However, groups can be
easily blocked from communicating through blocking of posts with the
given hashtags. We propose HooT, a system for censorship-resistant
microblogging. HooT presents an interface that is much like Twitter,
except that hashtags are replaced with very short hashes (e.g. 24 bits)
of the group identifier. Naturally, with such short hashes, hashtags
from different groups may collide and HooT users will actually seek to
create collisions. By encrypting all posts with keys derived from the
group identifiers, Hoot client software can filter out other groups'
posts while making such filtering difficult for the adversary. In
essence, by leveraging collisions, groups can tunnel their posts in
other groups' posts. A censor could not block a given group without also
blocking the other groups with colliding hashtags.
%
%HooT presents an interface that feels much like Twitter, except
%hashtags, which are normally used to denote conversation topics, are
%overloaded with cryptographic semantics; hashtags become cryptographic
%keys.  A HooT can then tunnel through other microblogging services,
%without observers able to read the message unless they know the original
%hashtag or tags that were used when the HooT was created.
%
We evaluate the feasibility of HooT through traces collected from
Twitter, showing that a single modern computer has enough computational
throughput to encrypt every tweet sent through Twitter in real time. We
also use these traces to analyze the bandwidth and anonymity tradeoffs
that would come with different variations on how group identifiers are
encoded and hashtags are selected to purposefully collide with one
another.
%
%Ultimately, censorship resistance and receiver anonymity comes from
%these collisions. A user subscribing to a high-volume innocuous hashtag
%would be indistinguishable from a user subscribing to a low-volume
%subversive hashtag.

\end{abstract}
