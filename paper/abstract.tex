\begin{abstract}

This paper presents the design and engineering of a system called HooT. HooT presents an
interface that feels much like Twitter, except hashtags, which are normally used to denote
conversation topics, are overloaded with cryptographic semantics; hashtags become cryptographic keys.
A HooT can then tunnel through other microblogging services, without observers able to read the
message unless they know the original hashtag or tags that were used when the HooT was created.
%
We analyze HooT through traces collected from the real Twitter, showing that a single modern computer 
has enough computational throughput to encrypt every tweet, in real time. Our traces also let us
analyze the bandwidth and anonymity tradeoffs that would come with different variations on how
plaintext hashtags are encoded and how they will purposefully collide with one another.
%
Ultimately, censorship resistance and receiver anonymity comes from these collisions. A user subscribing
to a high-volume innocuous hashtag would be indistinguishable from a user subscribing to a low-volume
subversive hashtag.


\hlfixme{TODO: This is an example of the hlfixme command. When we get
  down to the last 2-3 days, we'll set the fixme status from ``draft''
  to ``final,'' which will prevent compilation to pdf without fixing or
  removing all the fixmes.}

\end{abstract}
