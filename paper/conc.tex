\section{Conclusions}

To provide censorship resistance for groups wishing to communicate over a microblogging service like Twitter, we proposed \hoot. Hoots are private messages that are publicly posted but tagged with an identifier, so interested parties can find and decrypt them. By allowing many groups to collide with the same identifier, we protect recipient anonymity. We found that \hoot can be added to a service like Twitter with little additional computation resources and some increased bandwidth. By encouraging collisions, we require the user to memorize and transfer passwords which contain random characters. We identified a usability/entropy tradeoff with users lacking high entropy against a strong attacker. We believe that the amount of entropy is enough considering the weakness of the other components in the system like verbal key distribution. We can make it harder for an attacker, like hashing the Plain Tag 1,000 times instead of just once, but this only deters a strong attacker. In the future, a stronger key distribution system could remove the verbal component and allow for greater entropy and security.