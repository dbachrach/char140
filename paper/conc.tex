\section{Conclusions}

To provide censorship resistance and anonymity for groups wishing to communicate over a microblogging service like Twitter, we proposed \hoot. Hoots are private messages that are publicly posted but tagged with an identifier, allowing interested parties to efficiently find and decrypt them. By allowing many hash tags to collide with the same identifier, we protect recipient anonymity and use unrelated traffic as cover traffic. We found that \hoot can be added to a service like Twitter with little additional computational resources and reasonable additional bandwidth costs. We showed that users can have a experience that's virtually identical to a standard Twitter user, yet with radically better privacy. We also showed how it would be straightforward for Twitter to adopt \hoot and deploy it via web interfaces or via custom clients.
