\section{Background}

1) History of government censorship, man in the middle
	- Tunisia code injection
	- Chinese firewall
	- Crypto keys for important services to iranian source (Komodo)
	- Person providing netwrok (even over ssl) might be evil

2) Tor
	- Trying to work against government censorship
	
- Group Crypto Keys
	

\subsection{Definitions}

Privacy for groups and individuals in those groups has been investigated
in a variety of contexts. In this section, we describe how the aims of
Hoot relate to those of previously studied systems.

\paragraph{Anonymity} 
Systems for online anonymity, like Tor~\cite{tor} and
Mixminion~\cite{mixminion}, aim to protect users from being linked with
their traffic.

In this context, we can consider a variety of privacy attributes,
including:
\begin{itemize}
\item {\em sender anonymity}: The sender of a message is not
  identifiable from among a set of possible senders.
\item {\em recipient anonymity}: The recipient of a message is not
  identifiable from among a set of possible recipient.
\item {\em unlinkability}: Any of the items of interest (senders,
  recipients, or messages) cannot be linked with other items of
  interest.
\item {\em pseudonymity}: The use of pseudonyms as identifiers, either
  for a sender or a recipient.
\end{itemize}
Pfitzmann and Hansen have compiled a rich discussion of the meaning of
these and other terms~\cite{terminology}. Unlinkability is a rather
broad term. Two useful terms that can be derived from it are {\em
  recipient unlinkability} --- the recipient cannot be linked with the
sender(s) and messages --- and {\em relationship anonymity} --- the
sender and recipient cannot be linked with each other.

Hoot does not seek to provide sender anonymity: the adversary can
observe the fact that a given sender sent a particular message. Against
weaker adversaries who cannot eavesdrop on the sender, Hoot can provide
pseudonymity. More importantly, Hoot does aim to provide recipient

It does



Anonymity:
- sender anonymity
- Unlinkability
- Receiver anonymity
- subscriber anonymity
- Relationship anonymity
- Group anonymity

Social network privacy
- Community privacy
- Adversarial community discovery

Data and database privacy

Anti-censorship mechanisms:
- Eternity, Freehaven
- Freenet, Gnunet
- Publius, Tangler, Dagster
- Perng
- Serjantov

Metrics
- k-anonymity
- l-diversity
- differential privacy
- plausible deniability
- 

Techniques:
- Cover traffic
- Link cover
- Path cover
- RBC
- PIR/OT
 


%- Basic Crypto:
%	Message auth codes
%	Hash Functions
%	AES
%	Vanilla crypto
