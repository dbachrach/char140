\section{Background}

1) History of government censorship, man in the middle
	- Tunisia code injection
	- Chinese firewall
	- Crypto keys for important services to iranian source (Komodo)
	- Person providing netwrok (even over ssl) might be evil

2) Tor
	- Trying to work against government censorship
	
- Group Crypto Keys
	

\subsection{Definitions}

Privacy for groups and individuals in those groups has been investigated
in a variety of contexts. In this section, we describe how the aims of
Hoot relate to those of previously studied systems.

\paragraph{Anonymity} 
Systems for online anonymity, like Tor~\cite{tor} and
Mixminion~\cite{mixminion}, aim to protect users from being linked with
their traffic.

In this context, we can consider a variety of privacy attributes,
including:
\begin{itemize}
\item {\em sender anonymity}: The sender of a message is not
  identifiable from among a set of possible senders.
\item {\em recipient anonymity}: The recipient of a message is not
  identifiable from among a set of possible recipient.
\item {\em unlinkability}: Any of the items of interest (senders,
  recipients, or messages) cannot be linked with other items of
  interest.
\item {\em pseudonymity}: The use of pseudonyms as identifiers, either
  for a sender or a recipient.
\end{itemize}
Pfitzmann and Hansen have compiled a rich discussion of the meaning of
these and other terms~\cite{terminology}. Unlinkability is a rather
broad term. Two useful terms that can be derived from it are {\em
  recipient unlinkability} --- the recipient cannot be linked with the
sender(s) and messages --- and {\em relationship anonymity} --- the
sender and recipient cannot be linked with each other.

Hoot does not seek to provide sender anonymity: the adversary can
observe the fact that a given sender sent a particular message.
\hlfixme{is that true? I can tell that you sent a message on the bieber
  channel and that's it, right?}
%Against weaker adversaries who cannot eavesdrop on the sender, Hoot can
%provide pseudonymity.
Hoot does aim to provide recipient anonymity. In particular, we can say
that the main goal of Hoot is provide recipient unlinkability. We note
that it will be obvious to our attacker that the recipients receive the
messages; what is unclear is whether the receipients have the key to
decrypt and read the messages or if they are listening to another
channel.

Hoot can also be said to provide {\em subscriber anonymity}, as
introduced by Mislove et al. in their description of
AP3~\cite{ap3}. Hordes~\cite{hordes} and P5~\cite{P5} have similar
requirements. The main additional feature sought from subscriber anonymity
over recipient anonymity is that the act of subscribing should not
reveal information that could be used to break recipient anonymity.


- Group anonymity

Social network privacy
- Community privacy
- Adversarial community discovery

Data and database privacy

Anti-censorship mechanisms:
- Eternity, Freehaven
- Freenet, Gnunet
- Publius, Tangler, Dagster
- Perng
- Serjantov

Metrics
- k-anonymity
- l-diversity
- differential privacy
- plausible deniability
- 

Techniques:
- Cover traffic
- Link cover
- Path cover
- RBC
- PIR/OT
 


%- Basic Crypto:
%	Message auth codes
%	Hash Functions
%	AES
%	Vanilla crypto
