\section{Related Work}
\label{sec:related}

Censorship resistance has been carefully investigated in the context of
publishing documents and
file-sharing~\cite{eternity,freehaven,freenet,gnunet-esed,publius,tangler,dagster}.
Anderson proposed the Eternity Service, which would make documents
available for download and not allow any party to delete any
document~\cite{eternity}. Censorship resistance is provided by
replicating each document over many servers across many legal
jurisdictions. By anonymizing the system's communications, the service
would prevent linking plaintext files to the encrypted versions stored
on the servers by those who did not have the decryption key. Free
Haven~\cite{freehaven}, FreeNet~\cite{freenet}, and
GNUnet~\cite{gnunet-esed} provide similar properties in the context of
peer-to-peer file-sharing, as well as attempting to hide which peers are
hosting which files. Publius~\cite{publius} extends these approaches by
employing Shamir secret sharing~\cite{shamir} to make it harder to
determine what each server is storing. Tangler~\cite{tangler} and
Dagster~\cite{dagster} cryptographically intertwine data from different
documents in such a way that the censor can only force the system to
delete controversial documents by deleting ``legitimate'' documents and
thereby degrading the system as a whole. This provides a
censorship-resistance property similar to one provided by \hoot:
prevention of fine-grained censorship rather than prevention of
heavy-handed censorship.

Censorship resistance has also been studied in the context of
communication more broadly. Some systems aim to evade automated
filters. Feamster et al. propose Infranet, which passes information over
covert channels with the help of participating Web
servers~\cite{infranet}. In another work, Feamster et al. point out that
Infranet and other proxy-based solutions to censorship evasion face the
problem of finding the proxies~\cite{feamster03proxy}. To address this
problem, Feamster et al. require clients to solve cryptographic puzzles
to find a proxy. The Tor anonymity system faces a similar problem. It
has a widely-distributed list of servers~\cite{tor} and thus the censor
could block Tor by blocking access to the servers on the list. Tor uses
{\em bridges}, nodes that allow users to connect to Tor through them, to
evade such blocking~\cite{tor-bridges}. Although the bridges are not
published as widely, they must be disseminated to users and can be
discovered through the same channels by the censors. In particular,
China blocks access to Tor not only through blocking servers but bridges
as well~\cite{tor-china}.

In not relying on proxies, \hoot has the advantage of being easier to
use (since it doesn't require puzzles or CAPTCHAs) and of providing
plausible deniability, whereas the use of proxy-based censorship
resistance tools is likely to be inherently unacceptable to the censor
and could lead to the user being harassed or worse if found out.

\balance

%all groups the ability to use \hoot
%We argue that \hoot can be constructed so as to not 


%- Perng
%- Serjantov
%
%- Vasserman
