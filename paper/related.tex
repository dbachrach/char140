\section{Related Work}

Censorship resistance has been carefully investigated in the context of
publishing documents and
file-sharing~\cite{eternity,freehaven,freenet,gnunet-esed,publius,tangler,dagster}.
Anderson proposed the Eternity Service, which would make documents
available for download and not allow any party to delete any
document~\cite{eternity}. Censorship resistance is provided by
replicating each document over many servers across many legal
jurisdictions. By anonymizing the system's communications, the service
would prevent linking plaintext files to the encrypted versions stored
on the servers by those who did not have the decryption key. Free
Haven~\cite{freehaven}, FreeNet~\cite{freenet}, and
GNUnet~\cite{gnunet-esed} provide similar properties in the context of
peer-to-peer file-sharing, as well as attempting to hide which peers are
hosting which files. Publius~\cite{publius} extends these approaches by
employing Shamir secret sharing~\cite{shamir} to make it harder to
determine what each server is storing. Tangler~\cite{tangler} and
Dagster~\cite{dagster} cryptographically intertwine data from different
documents in such a way that the censor can only force the system to
delete controversial documents by deleting ``legitimate'' documents and
thereby degrading the system as a whole. This provides a
censorship-resistance property similar to one provided by \hoot:
prevention of fine-grained censorship rather than prevention of
heavy-handed censorship.

Censorship resistance has also been studied in the context of
communication more broadly

Communications technologies:
- Tor bridges


Infranet
- 

\balance

%all groups the ability to use \hoot
%We argue that \hoot can be constructed so as to not 


%- Perng
%- Serjantov
%
%- Vasserman
