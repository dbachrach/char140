\section{Implementation}
\label{sec:implementation}

In this section we describe our prototype \hoot implementation, which we
use for performance experiments (see
Section~\ref{sec:experiments}). Additionally, our discussion in this
section helps to illustrate the design choices and trade-offs available
in the \hoot approach.

%\subsection{Generating a Hoot}
\paragraphX{Generating a Hoot}
The \hoot protocol allows for a variety of different encryption,
hashing, and message authentication schemes. Our prototype client,
implemented in Python using the
PyCrypto\footnote{\url{http://www.dlitz.net/software/pycrypto/}}
library, takes the following steps to construct a \msg:
\begin{itemize}
\item We generate a long tag by taking a SHA2-256 hash of the plain
  tag. The first 128 bits of the long tag are used exclusively for the
  short tag. The remaining bits
  bits are used as the tag key, $k_{\mathrm{tag}}$, to encrypt the
  session keys.
\item Using PyCrypto's cryptographically-strong random number
  generator, we generate 
  a random 128-bit encryption key, $k_{\mathrm{enc}}$, and a random 128-bit MAC key,
  $k_{\mathrm{mac}}$. These keys are
  concatenated together and then encrypted with the tag key using AES in
  counter mode with a fixed initial counter of $0$. (The randomly
  chosen encryption key for AES ensures a suitable level of
  non-determinism in its ciphertext.)
\item We encrypt the plaintext message with $k_{\mathrm{enc}}$, again using AES in
  counter mode, and use $k_{\mathrm{mac}}$ to generate an HMAC-SHA1
  message authentication code over the encrypted plaintext of the
  message.
\item We print the \msg, consisting of a {\tt \#} symbol, the short tag,
  a space, the encrypted keys, the HMAC digest, and the ciphertext.
\end{itemize}

% We note that our prototype does not currently include a mechanism for
% generating collisions between short tags.
% This would be useful primarily
% for the first person to start the group or generate a new plain tag
% (effectively resetting the password).
(We describe a prototype of \proc{Find-Tag} in
Section~\ref{sec:collider}.)

%To generate collisions, the client should accept as input an existing
%short tag that the user wants her tag to collide with and the prefix to
%be used for the plain tag (rather than the entire plain tag). The
%client could then apply a random sequence of suffix values to the
%prefix, take hashes of the resulting plain tags, and compare the short
%tag derived from each hash with the desired short tag until it finds
%match found. We address the cost to find such a collision in
%Section~\ref{sec:experiments}.

\paragraphX{Message length.}  
We wish to render \msgs in a format that can be transmitted via
Twitter. The primarily difficulty we face is Twitter's 140 character
limit. We must also ensure that the short tags are rendered in standard
Twitter hashtag format (i.e., preceded with a {\tt \#} character and
followed by whitespace) such that standard Twitter searching mechanisms
will efficiently find them.

Interestingly, Twitter has a very broad definition of a character. Based
on our testing, we believe that Twitter limits tweets to 140 Unicode
(UTF-8) glyphs. While we could certainly take advantage of this to
squeeze the longest possible \msgs into a single tweet, particularly if
we were willing to restrict plaintext messages to 7-bit ASCII, we chose
not to pursue this for our initial implementation. Instead, we went with
a standard Base64 encoding (the letters A-Z, a-z, 0-9, +, and /),
yielding only six bits per glyph.

Assuming a single short tag of two Base64 glyphs (12 bits), the maximum
plain text message with our prototype implementation would be 31
single-byte characters. (We would need 79 glyphs to represent the
message header, including one short tag, leaving 61 glyphs for the
message, which could then be at most 45 bytes of plain text.)

If, however, we were to implement a more efficient Unicode packing, we
could certainly do much better. UTF-8 allows for just over 1 million
values in a single glyph, not all of which are currently in use\footnote{Wikipedia has a
  reasonably good discussion on this topic:
  \url{http://en.wikipedia.org/wiki/UTF-8}.}. As such, a Unicode packer
should be able to achieve 20 bits per glyph. With this, the entire \hoot
header, with one short tag, could be encoded in 26 glyphs, leaving 114
glyphs for the ciphertext. For users used to Twitter's 140-character
length restriction,
% which also covers the space available for hashtags
% that are generally not necessary given the short tag, 
this is likely to be reasonable (moreso if they limit themselves to
7-bit ASCII characters). Of course, we could also build any
compression scheme into the \hoot protocol, perhaps adding a 
few bits to the header to indicate a language group, and thus
initialize the compression system with a corpus of common words
and phrases. This would radically improve the efficiency of the
compression scheme and thus the amount of data that could be encoded
in a \hoot, while still respecting Twitter's 140-character maximum.

We note that Twitter is not the only microblogging system that we could
leverage for \hoot traffic. \hoot could just as easily be built atop
Google Buzz, which doesn't have Twitter's 140 character
limit.\footnote{The 140 character limit is an artifact of Twitter's
  original intent to support SMS cellular telephone messaging as a way
  of delivering tweets. This unfortunate design decision was originally
  made in 1985 and still haunts us today~\cite{latimes-char160}.}
