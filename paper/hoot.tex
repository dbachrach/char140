% hoot.tex

% Author: Dustin Bachrach, Christopher Nunu, Dan Wallach, Matt Wright

% Revisions:  28 March 2011

\documentclass{acm_proc_article-sp}

\usepackage{mathptm}
\usepackage{times}

\begin{document}

\numberofauthors{4}

\author{
\alignauthor
	Dustin Bachrach\\
	\affaddr{Department of Computer Science}\\
	\affaddr{Rice University}\\
	\affaddr{Houston, Texas}\\
	\email{ahdustin@gmail.com}	
	
\alignauthor
 	Christopher Nunu\\
	\affaddr{Department of Computer Science}\\
	\affaddr{Rice University}\\
	\affaddr{Houston, Texas}\\
	\email{canunu@gmail.com}
	
\alignauthor
	Dan Wallach\\
	\affaddr{Department of Computer Science}\\
	\affaddr{Rice University}\\
	\affaddr{Houston, Texas}\\
	\email{dwallach@rice.edu}\\

\and
\alignauthor
	Matt Wright\\
	\affaddr{Computer Science and Engineering}\\
	\affaddr{University of Texas at Arlington}\\
	\affaddr{Arlington, Texas}\\
	\email{mwright@cse.uta.edu}
}

\title{Hoot}
\date{May 6th, 2011}

\maketitle

\begin{abstract}

\end{abstract}

\section{Introduction}

- Privacy

- Define anonymity

--- Subscriber anonymity

--- Sender anonymity

- What we are doing
-- Define a protocol for users to communicate over an insecure public network like twitter with message confidentiality and subscriber anonymity. 

- Punting on plain tag distribution

\section{Related Work}

- Todo

\section{Design}

- Confidentiality of a message

- Identification of a message

- Collisions

-- Induced collisions

-- Trade off between more or less collisions

- 2 Session keys
-- Session Key (Encryption)
-- HMAC Key (Integrity)

- What a message looks like

\section{Implementation}

- Protocol

-- Session Keys
-- Encryption Session Key: Rand (16 bits)
-- Integrity Session Key: Rand (20 bits)

-- Identification: Sha256 plain tag
--- first 128 bits are used for long tag
--- first n bytes of long tag become short tag

-- Confidentiality: AES (using session key)
-- HMAC: SHA (integrity key)
-- 

\section{Incremental Deployment}

\subsection{Message Length}

- Two encoding options
-- Base64
	-- Simple
	-- Universal
	-- Longer message size 
		---m + 20 * (n/16) + 118 
		---m <= 2, n <= 31, yields Hoot <= 140 characters
-- Unicode
	-- Describe Twitter's counting of `characters'
	-- Use to our advantage
	-- Allows normal English character tweets to fit in 140 chars
	-- Shorter message size
	--- 64 + m + 12 * (n / 16)
	--- m <= 2, n <= 111, yields Hoot <= 140 characters

\subsection{Pipeline Performance}

- Twitter's stats (http://blog.twitter.com/2011/03/numbers.html):
	-- 140 million tpd
	-max: 6939 hps
	
- All our numbers are using base64 encoding:

Encryption
----------
100,000 hoot encryptions -> 27.631s        3619.123 hps
500,000 hoot encryptions - 137.402s        3638.957 hps
Avg: 3614.531 hps

Decryption
----------
100,000 hoot decryptions -> 6.430s         15552.099 hps
500,000 hoot decryptions - 32.005s         15622.558 hps
Avg: 15587.328 hps

- Given these numbers, a Twitter server could easily encrypt the entire twitter firehose.

- In reality, would probably get less tweets, but this gives us current upper bound

- To be certain that twitter can keep up, add a few nodes and optimize, and will definitely keep up.

- Interesting to note that decryption rate is important on client, since a client is searching tweets to find potential hoots (and may not collaborate w/ twitter)



\section{Discussion}

- would twitter actually do this
	-- Probably not
	-- Twitter wants to know what people are talking about
	-- Honestly, best if twitter doesn't do this, since users can't ``trust'' twitter, encryption is better done on their own machine and results posted to twitter
- use of this protocol (or who would use it)

\section{Conclusions}



\bibliographystyle{abbrv}
\bibliography{hoot–bib}

\balancecolumns{}

\end{document}
