\section{Discussion} \label{sec:discuss}

In this section, we discuss a variety of issues and future extensions of the Hoot design.

\subsection{Incremental Rollout}

\hl{
1. Can you incrementally roll it out
	- Service provided by twitter or yourself
2. 140 character limit
	- Twitter doesn't have to go too far to have all the metadata for encryption
	
	- How hard is it for twitter to do this. 
	- Reference message fitting into 140 char using unicode
	- Reference peak hps, and how our system holds up
}
\subsection{Adoption}	

The next question to ask after we have shown that rolling out the Hoot service is feasible would be whether Twitter would actually implement such a feature. Based on the nature of Twitter, we believe that Twitter would not add a secure messaging framework like Hoot. Twitter as a company needs to know what people are talking about, so it can provide relevant advertisement. Adding the Hoot infrastructure to Twitter would prevent Twitter from knowing the content of the messages, and so we believe the Hoot service will never be adopted. However, this need not prevent individuals from using the Hoot protocol over Twitter. As long as Twitter faithfully delivers tweets, users are free to run the Hoot encryption on their own machines over their messages and then post the output to Twitter. In fact, this method is more secure in that a user only has to trust their machine. If Twitter is responsible for encrypting a message, nothing is stopping them from keeping a copy of the plain text message. Paranoid users will always want to encrypt messages themselves, so Twitter adopting the Hoot protocol is unnecessary.
		
\subsection{Usability}

As described in the Cover Traffic section, a group can deliberately collide with a popular tag by concatenating an easily rememberable string of text with random letters or numbers. As Miller\cite{miller56} noted, people can remember  
7 +/- 2 unique packets of data, so as long as these collisions can be generated with a suffix in that range a user is likely to be able to remember it. We could further improve rememberability by restricting suffix values to only digits, and then generate a suffix of 7 or 10 digits, emulating a phone number. The main requirement is that a group should be able to have a unique shared secret that can deliberately collide with other tags while still being easy to remember and more importantly easy to transfer. Since our proposal does not deal with key transfer, we assume our key communication is done through whisper channels and thus must be short and memorizable for simple transfer. We have found that a collision can be found with only appending 3-6 characters to a prefix, so deliberate collisions can both be found and transferred without much effort.

\hl{
-- Deliberate posting crap (to hide your stuff).
-- Increases work factor (signal to noise)
-- Fairly strong attacker, but user who don't have ability for a lot of entropy
-- 70 bits is enough compared to the weakness of everything else
-- Discuss how we can make it harder for an attacker. Maybe hashing 1000 times...
}

\hlfxnote{ Having the client (+ Twitter?) find some groups for you to
  collide with, rather than you having to find them yourself.}

%For some users, it may be appropriate for the client to search twitter
%characteristics of the groups with which the user wants the collision,
%e.g. the activity level of the groups and

\subsection{Adaptability}

Since Hoot benefits from and encourages collision, there will come a point when there is too much collision, i.e. there will be too much noise to signal for a specific group when following an identifier that collides with many groups. At some point, the entire Hoot universe will want to decrease the amount of collisions. The simple solution is to increase the Short Tag length. Whenever there is too many collisions, the Short Tag length will be increased by one, which will result in far less collisions until the communication increases and eventually the cycle continues. 

The nature of Hoot makes it convenient for this Short Tag length tuning to be done at the system level or at the group level. For instance, a group that is overrun with collisions with popular group could simply increase their Short Tag length internally. All Hoots would include the extra character in the Short Tag, and all subscribes would search for the longer Short Tag. It is also as simple for the entire system to switch to a longer identifier by having an ``oracle'' broadcast to everyone that the system Short Tag length has increased.

Even in a distributed system where some users may not receive the oracle's broadcast, the system still functions. For example, assume user $A$ is following a group who's Long Tag begins ``A5trxq...'', and in $A$'s view, Short Tags should be 2 characters. For user $B$, who posts messages to the group, the Short Tag length should be 3. $B$'s Hoot would then look like ``\#A5t'', but $A$ would be searching for ``\#A5". Twitter's search functionality will still return $B$'s messages for $A$'s search. Although $A$ will be getting much more noise than signal when they search at the shorter length, they will not miss any Hoots.

Previously, we stated that a Short Tag is the first K bytes of the Long Tag, so we explicitly put a limit onto how long a Short Tag can be. Does this prevent us from adapting to an ever growing Hoot ecosystem? In our implementation, we use the first 16 bytes of the Long Tag for identification. Therefore using Base64 encoding, we get $62^{16}$ or $4\times10^{28}$ unique identifiers. In a system with 16 byte Short Tags, there would rarely if ever be collisions, much less too many collisions to increase the length.

Finally, as a result of the encoding systems, the size of the Short Tag space increases by a large factor for each character added. This takes a space that is highly populated to a space that is very sparse, so it is very possible that by adding one character to tag identifiers that groups would be uniquely identified. Groups that also deliberately collided with another group would often not collide at a longer Short Tag length. It would be convenient to instead grow the space by single bits at a time, so as to double the space, a much smaller factor. Unfortunately, Twitter is a character driven environment rather than bit driven, so this is difficult to address. It also does not deal with deliberate collisions. All deliberate collisions will need to be regenerated after a Short Tag length increase.


\hl{
chris 
- need a citation for core growth so we can make argument about whether searching will keep being feasible
- twitter vs moore
- need a citation, or argument, for how fast we expect twitter to grow in the future, can we turn the nob slow enough to keep in pace with processing power, and still be usable? 
}

\subsection{Alternate Backends}

Even though our protocol was designed with Twitter in mind, it is extensible to other systems and platforms. The Hoot protocol describes a secure way to transfer short messages (with short encryption overhead) across a publicly available network completely openly. Twitter is a great example of this environment but not the only one. Hoot could interact, for example, with a Distributed Hash Table system (DHT) like CHORD or Pastry. Importantly, Hoot can interface with both centralized systems like Twitter or Buzz or decentralized systems. As long as the platform's content is publicly searchable, the Hoot protocol allows secure transmission to anonymous group followers.
