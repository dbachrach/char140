\section{Discussion} \label{sec:discuss}

In this section, we discuss a variety of issues and future extensions of the \hoot design.

\subsection{Deployment}

There are two different paths through which \hoot could be deployed. First, it would be straightforward to implement a proxy service, whether operated by Twitter or by an independent third party, that reads every tweet, encrypts it, and provides a Twitter-like interface to the stream of \hoot messages. This would incur non-trivial monetary costs for the bandwidth and computation resources, but it's still well within the means of many organizations. A deeper concern is that any country could simply block traffic to or from the \hoot proxy server.

Alternately, the proxy server could read every normal tweet, encrypt it, and post it back to Twitter. This, naturally, would double the number of tweets, which makes it something that Twitter would probably insist on doing themselves rather than accepting from a third-party service. A Twitter-internal implementation might instead, for example, precompute the short tags for every tweet but regenerate \hoot-encrypted messages on the fly rather than storing two copies.

Any strategy where \hoot messages are injected by a proxy has the property that the \hoot need not necessarily specify the original sender's Twitter username.  Breaking the binding between the sender and the public \hoot message would greatly improve sender anonymity, but it would also allow any user who knows the relevant plain tag to impersonate any other member. We could imagine a variety of ways to add some sort of digital signature or hash-chaining layer into the \hoot message format to differentiate users with ``read-only'' versus ``posting'' privileges. We leave this for future work.

Regardless, a critical question is whether the plain tag for a \hoot message should ever be sent to Twitter. For users with dedicated clients, the users' plain tags could be stored locally, avoiding any issue where a compromise of Twitter's server becomes a single point of failure for \hoot's confidentiality and integrity. This would clearly be the preferred modality for \hoot usage, but that may not be feasible for a large number of potential \hoot users.

It's important to note that Twitter's web interface offers full SSL encryption. Twitter's dedicated smartphone clients currently use OAuth signatures, protecting message integrity but not privacy. Full encryption would be desirable for smartphone clients to defeat active adversaries who might use deep packet inspection to detect \hoot messages in transit and close the connection.

A few other considerations are worth noting:
\begin{itemize}
\item If Twitter could be convinced to directly support \hoot, then they could certainly stretch the 140-character limit to allow \hoot plaintext messages to be as long as regular Twitter messages.

\item If \hoot users were connected to Twitter's web interface, then it would be essential for \hoot messages to be decrypted in the browser, via client-side JavaScript, rather than on the server. We want every \hoot message matching a short tag query to be transmitted over the network to ensure that passive network eavesdroppers, even in the face of full SSL encryption, will only see traffic patterns corresponding to particular short tags, which leaks nothing useful, versus particular plain tags, which would eliminate any cover traffic and thus leak valuable information to the adversary.

\item Regardless, an adversary conducting traffic analysis might be able to distinguish a subscription to a \hoot short tag from a subscription to something else, solely based on messages' timing and size. This could be addressed by having {\em all} Twitter users receive \hoot cover traffic for a randomly selected \hoot short tag, which their clients would then ignore. It could also be addressed by hacking normal Twitter clients from unsuspecting users to treat any request to follow a hashtag by having them follow the corresponding \hoot short tag, instead.

\item At some point, Twitter may deploy content-aware advertising on its services. Advertising should never be allowed to operate based on a decrypted \hoot, as the advertiser could well be the adversary, and could possibly use analysis of advertisement metrics to violate \hoot users' privacy.

\item Twitter may not have any interest in \hoot or anything like it. See Section~\ref{sec:dht} for a discussion of alternative backend designs.

\end{itemize}



\if 0
Twitter would not need to go far to have all the metadata necessary to encrypt and decrypt for their users. They would need an interface to allow users to mark a tweet as encrypted, and everything in the tweet would bee encrypted using the hashtag in the message as the plain tag. As we showed previously, fitting into the 140 character limit is possible with Unicode encryption, but Twitter could easily make an internal exception for \hoot. We also showed that computation for encryption and decryption is negligible. One computer can do it, and Twitter has entire racks of computers. The real issue is how the \hoot system would affect bandwidth for Twitter. The bandwidth would increase based on the group collisions and the number of Hoots to a group. Users no longer follow a single tag and read everything. They subscribe to a Short Tag, which collides with many other groups, and therefore they must see all the noise of the other groups on their Short Tag. Fundamentally, Twitter has to pay for the receiver anonymity inherent in the \hoot protocol. 
% mkw -- looks okay. 
%\hlfixme{Someone double check this reasoning:} 
For $N$ groups who collide onto $S$ Short Tags with each group posting $M$ messages, the added bandwidth would be $M\frac{N}{S}$ as opposed to the original bandwidth of $M$. This accounts only for group messages, when in reality there are many more tweets that do not have hashtags. The bandwidth additions for \hoot might be very small compared to the bandwidth used to send ordinary non-group messages.

According to Sandler and Wallach~\cite{sandler09}, Twitter users, on average, send 100 messages and have 100 followers. It is difficult to know how many groups (hashtags) users post to, but it is even more difficult to know how many users listen to a particular group. After Sen. Jon Kyl made an inaccurate statement, his spokesperson cleared it up by saying it was ``not intended to be a factual statement''~\cite{politico11}. Stephen Colbert then encouraged his followers to tweet using the hashtag \#NotIntendedToBeAFactualStatement. This quickly became a popular Twitter group, but it is impossible to know how many people were actually listening to the group but not posting. Even those posting to the group might not be listening to all the other tweets to the group. We could treat the number of posters and the number of listeners as equivalent, but that is not intended to be a factual statement.

Regardless of the number of groups or Short Tags, Twitter's bandwidth bill would certainly go up, and since their finances are already dubious, implementing \hoot may be a concern.

\subsection{Adoption}	

The next question to ask after we have shown that rolling out the Hoot service is feasible would be whether Twitter would actually implement such a feature. Based on the nature of Twitter, we believe that Twitter would not add a secure messaging framework like Hoot. Twitter as a company needs to know what people are talking about, so it can provide relevant advertisement. Adding the Hoot infrastructure to Twitter would prevent Twitter from knowing the content of the messages, and so we believe the Hoot service will never be adopted. However, this need not prevent individuals from using the Hoot protocol over Twitter. As long as Twitter faithfully delivers tweets, users are free to run the Hoot encryption on their own machines over their messages and then post the output to Twitter. In fact, this method is more secure in that a user only has to trust their machine. If Twitter is responsible for encrypting a message, nothing is stopping them from keeping a copy of the plain text message. Paranoid users will always want to encrypt messages themselves, so Twitter adopting the Hoot protocol is unnecessary.
\fi
		
\subsection{Usability}

In essence, \hoot requires its users to manually negotiate a group cryptographic key management system. Even though the use case for \hoot looks and feels much like ``normal'' Twitter hashtags, we are still putting a non-trivial amount of trust into a protocol that requires users to use literal gossip to spread key material. Even highly motivated users could well make mistakes, and any small error would result in the inability to decrypt the desired \hoot messages. In particular, a significant amount of entropy is required for adequate security, causing \hoot plain tags to push the boundaries on human memorability. In effect, a good \hoot plain tag is comparable to a strong account password. \hoot's use of machine-generated plain tags would have comparable issues with organizations that require users to have strong passwords that are never written down and frequently changed.  And, unlike such organizations which can adopt a variety of ``two factor'' authentication technologies to reduce their reliance on passwords, \hoot fundamentally needs plain tags as strong as strong user passwords.

About the only good news in this process is the growing ubiquity of smartphones, typically having a variety of methods to communicate with other phones nearby. This would allow a set of plain tags to be quickly and painlessly shared via means including two-dimensional barcodes (displayed on one phone's screen, read with another phone's camera), a variety of close-range networking technologies (near-field communication, Bluetooth, ad-hoc WiFi, or infrared file transfer), or even acoustic transfer from one phone's speaker to another phone's microphone. If, in fact, the predominant modality for \hoot plain tag sharing is one of these mechanisms, then plaintag memorizability becomes a non-issue. Plain tags can then be implemented with general-purpose securely-chosen random numbers.

A related usability question is the process for selecting the desired external hash tag with which to collide a \hoot plain tag. Earlier in this paper, we cavalierly suggested that famous pop singers helpfully provide all the cover traffic we might ever want, but this process ultimately needs to be handled with care, since recipient deniability requires the human recipient to have a convincing story under coercive pressure, and it's certainly the case that many people cannot convincingly demonstrate their admiration for an overseas pop sensation. Ultimately, the selection of a suitable Twitter hashtag for \hoot collisions is a process that will benefit from a dedicated web site, engineered for this specific purpose. Naturally, a censorship-happy country will do its best to block access to such a web site, requiring group organizers to leverage Tor or otherwise be resourceful enough to connect to the outside world.

\if 0

As we described in the Cover Traffic section, a group can deliberately collide with a popular tag by concatenating an easily rememberable string of text with random letters or numbers. As Miller~\cite{miller56} noted, people can remember  
7 +/- 2 unique packets of data, so as long as these collisions can be generated with a suffix in that range a user is likely to be able to remember it. We could further improve rememberability by restricting suffix values to only digits, and then generate a suffix of 7 or 10 digits, emulating a phone number. The main requirement is that a group should be able to have a unique shared secret that can deliberately collide with other tags while still being easy to remember and more importantly easy to transfer. Since our proposal does not deal with key transfer, we assume our key communication is done through whisper channels and thus must be short and memorizable for simple transfer. We have found that a collision can be found with only appending 3-6 characters to a prefix, so deliberate collisions can both be found and transferred without much effort. By brute-forcing to find a collision, the string of characters after the prefix  can be considered "random." Yan et. al~\cite{yan00} discussed random and mnemonic passwords. Our string of characters will rarely be mnemonic, so it suffers from the issues they mentioned like users forgetting the password, but on the other hand random passwords are stronger than mnemonics as shown by Kuo et. al.~\cite{passwords06}.

To make the system even more usable, the client or Twitter should automatically find groups for the user to collide with rather than the user finding it by hand. This system could even allow the user to tweak a parameter for how active a group to collide with.
\fi


%For some users, it may be appropriate for the client to search twitter
%characteristics of the groups with which the user wants the collision,
%e.g. the activity level of the groups and

\subsection{Adaptability and scalability}

Since \hoot benefits from and encourages collisions in the short tag space, there will come a point when there are simply too many collision, i.e., the ratio of desired messages to cover traffic will eventually become vanishingly small. The seemingly obvious solution is to adjust the length of short tags. Doubling the number of possible short tags reduces the number of collisions in half, or at least it would if there were  a uniform distribution of raw traffic across the space of short tags. Given how a small number of hashtags can become hot, trending topics, it's far from obvious how changing the standard short tag length would impact the distribution of message, nevermind the pragmatic concerns with keeping every user on the same page when the short tag length is supposed to change.

This process would benefit from the active engagement of group organizers. Consider the case where a low-volume group organizers selects a low-volume plain tag with which to collide. Everything works great until some heretofore unseen hashtag suddenly rockets to the top of Twitter's trending discussions and just happens to collide with the very same plain tag. Our group organizers would need to stay on top of these issues, with a migration strategy ready to go. Option \#1: implement a command to migrate to a new plain tag which could be broadcast in the message body of a \hoot; this would, of course, require some way of authenticating the command. Option \#2: group organizers would give out multiple plain tags, far in advance, allowing them to issue a suggestion that the group is moving from plain tag \#1 to plain tag \#2.

\if 0
messages  there will be too much noise to signal for a specific group when following an identifier that collides with many groups. At some point, the entire Hoot universe will want to decrease the amount of collisions. The simple solution is to increase the Short Tag length. Whenever there is too many collisions, the Short Tag length will be increased by one, which will result in far less collisions until the communication increases and eventually the cycle continues. 

The nature of Hoot makes it convenient for this Short Tag length tuning to be done at the system level or at the group level. For instance, a group that is overrun with collisions with popular group could simply increase their Short Tag length internally. All Hoots would include the extra character in the Short Tag, and all subscribes would search for the longer Short Tag. It is also as simple for the entire system to switch to a longer identifier by having an ``oracle'' broadcast to everyone that the system Short Tag length has increased.

Even in a distributed system where some users may not receive the oracle's broadcast, the system still functions. For example, assume user $A$ is following a group who's Long Tag begins ``A5trxq...'', and in $A$'s view, Short Tags should be 2 characters. For user $B$, who posts messages to the group, the Short Tag length should be 3. $B$'s Hoot would then look like ``\#A5t'', but $A$ would be searching for ``\#A5". Twitter's search functionality will still return $B$'s messages for $A$'s search. Although $A$ will be getting much more noise than signal when they search at the shorter length, they will not miss any Hoots.

Previously, we stated that a Short Tag is the first K bytes of the Long Tag, so we explicitly put a limit onto how long a Short Tag can be. Does this prevent us from adapting to an ever growing Hoot ecosystem? In our implementation, we use the first 16 bytes of the Long Tag for identification. Therefore using Base64 encoding, we get $62^{16}$ or $4\times10^{28}$ unique identifiers. In a system with 16 byte Short Tags, there would rarely if ever be collisions, much less too many collisions to increase the length.

Finally, as a result of the encoding systems, the size of the Short Tag space increases by a large factor for each character added. This takes a space that is highly populated to a space that is very sparse, so it is very possible that by adding one character to tag identifiers that groups would be uniquely identified. Groups that also deliberately collided with another group would often not collide at a longer Short Tag length. It would be convenient to instead grow the space by single bits at a time, so as to double the space, a much smaller factor. Unfortunately, Twitter is a character driven environment rather than bit driven, so this is difficult to address. It also does not deal with deliberate collisions. All deliberate collisions will need to be regenerated after a Short Tag length increase.
\fi


% \hl{
% chris 
% - need a citation for core growth so we can make argument about whether searching will keep being feasible
% - twitter vs moore
% - need a citation, or argument, for how fast we expect twitter to grow in the future, can we turn the nob slow enough to keep in pace with processing power, and still be usable? 
% }

\subsection{Alternate backends}
\label{sec:dht}

Even though our protocol was designed with Twitter in mind, it is extensible to other systems and platforms. The \hoot protocol describes a secure way to transfer short messages (with low encryption overhead) across virtually any publicly available content distribution network. All that \hoot requires is efficient search primitives for each message's short tag or tags. Everything else is handled by client-side software.

Consequently, if Twitter had no interest in \hoot or concluded that it would not be supportable, then \hoot could just as easily work with a variety of centralized or distributed network services. To pick one possibility, \hoot could use the BitTorrent ``distributed tracker'', or any other large and public distributed hash table (DHT) service, to store messages with the short tags used to name the DHT nodes responsible for their storage. FeedTree~\cite{sandler05feedtree} is one of many systems that have attempted to support micropublishing on a DHT. (Other DHT-based ideas are discussed in Section~\ref{sec:related}.) 
