\section{Introduction}

Recent events in Egypt, Tunisia, and many other countries have shown the social networking sites (Facebook, Twitter, and presumably others) played a non-trivial role in helping people organize themselves and plan social protests. Egypt was notable in that they eventually cut themselves off from the entire Internet, in a belated and ultimately ineffectual attempt to turn the tide of social protest. While it's difficult to draw overarching conclusions from these few data points, it is clear that social media is tolerated in a number of countries who may well turn it off, in its entirety, when faced with internal turmoil.

As a first step toward improving social network systems for such environments, we wish to design a system to enable the use of strong cryptographic primitives, overlaid on existing microblogging services like Twitter. We wish to enable secure group communication without requiring prearranged public key hierarchies. We wish to provide some measure of anonymity, by blurring the binding between the sender and recipients of any given message, using existing innocuous messages as a form of cover traffic, providing some measure of deniability, at least for recipients of messages.

In particular, our work aims to present an interface where most users never see or concern themselves with cryptographic keys. Instead, one of our key insights is that we can overload Twitter's ``hashtag'' mechanism as a way of deriving cryptographic key material. Before we can sort out exactly how that should work, it's important to first understand how hashtags are used in the wild.

Hashtags are widely used in the Twitter universe to label topics which others will then subscribe to and follow. For example, most Usenix conferences adopt the tag {\tt \#usenix}, allowing attendees to discuss the conference with one another in real time. Political protests might end up using several different tags (e.g., Egyptian discussions happen under {\tt \#tahrir}, {\tt \#jan25}, {\tt \#25jan}, and {\tt \#egypt}, among others). Hashtags searches are generally case-insensitive.

Some tags have staggering volumes of messages. To pick a notable example, pop singer Justin Bieber asked his roughly 9 million followers to discuss his movie, {\em Never Say Never} using the hashtag {\tt \#nsn3d}. At its peak, roughly 1\% of Twitter's traffic mentioned this tag\footnote{Statistics via Trendistic, Topsy, and HashTracking.}. Since the movie's release in February 2011, there have been roughly 164 thousand tweets using {\tt \#nsn3d}, an average of 1.6 per minute with significantly higher peaks. 500 recent tweets generated 346 thousand impressions, reaching an audience of 212 thousand followers within a 24 hour period (measured in mid-April 2011).

If a group is communicating in secret, not only do we want to
protect the content of the communication, but we would like to conceal
subscription to a particular group. For example, a rebellion group
wishes to communicate over Twitter using HooTs, but it can be
dangerous for a supporter of the rebellion to listen and subscribe to
the identifier of the group. If, however, the identifier of the group
can collide with the identifier of a popular Internet topic, like the aforementioned
Justin Bieber or many other such Twitter celebrities, group followers
can shadow their rebellious activities with other innocent topics.

Of course, not all tags are as popular. We will show later, in Section~\ref{sec:experiments}, that hashtag usage follows a power-law distribution; a small number of hashtags are incredibly widely used and large numbers of tags are used very rarely or only once. Consequently, for any design that attempts to convert hashtags to cryptographic keys, we will want to give some amount of control over to the person selecting a new tag, e.g., allowing the organizers of a particular protest to either deliberately collide with a popular tag, like Bieber's {\tt \#nsn3d}, or to deliberately avoid it.

Ultimately, we see two main paths to designing our system. One option would be to send encrypted messages that include the real {\tt \#nsn3d} tag, perhaps engineering some sort of steganographic process that tries to hide the plaintext within messages that are statistically similar to other posts from Bieber's fan, but it seems inappropriate to produce false messages like this. The other possibility is to imagine that {\em all} Twitter messages are encrypted in a uniform way, where knowing the plaintext of the hashtags would enable the decryption of a message. (It's easy to see a proxy server, of some sort, providing an ``encrypted'' interface to Twitter in this fashion; more on that in Section~\ref{sec:whatever-else}.) This is the design we chose to pursue. In this setup, we can encrypt and MAC every message with a random session key, which can be decrypted if the user knows the proper hashtag. ``Encrypted'' hashtags can also be generated by hashing the plaintext hashtags and truncating those hashes (we address this unwieldy vocabulary when we present our design in Section~\ref{sec:design}). Consequently, two different plaintext hashtags can collide with each other with a probability related to the number of bits in the truncated hash.

\if 0
% decided not to use this, and instead convert "threat model" into "design goals"
To make a system like this ``real'', we must:
\begin{itemize}
\item Ensure that real Twitter messages use enough hashtags that, when reflected through our system, provide a significant amount of cover traffic in which to hide other messages.
\item Ensure that hashtags, when used directly as secrets, can be long enough to defeat computational brute force searches, yet be short enough to be memorized and passed along through spoken gossip.
\item Ensure that followers of secret hashtags have a defensible cover story (e.g., ``I'm just a big fan of Justin Bieber!'').
\item Ensure that those who post with secret hashtags can protect themselves from discovery.
\item Ensure that censorship systems, should they not know the secret hashtags, cannot distinguish those messages from other perfectly legitimate messages. We want to ensure that the only surefire way to filter secret messages is to disable the entire social network.
\item Be backward compatible, to the extent possible, with the real Twitter.
\item Minimize the extent to which we need to leverage external anonymity/censorship-resistance systems.
\end{itemize}
\fi

% Our work aims to offer modest improvements to the ability for groups of people to carry out conversations, via social media, 
% 
% \hl{
% - Problem: Twitter like semantics w/ encrypted messages
% 	- Follow a Hash tag
% 	- Take hash tag and create something with crypto strength
% 	- Something derived from tags you can search on
% 	- But also deliberate collisions (cover traffic)
% 	
% - Like to have thing that feels like twitter but anonymity properties:
% 
% - Twitter/Facebook relevant in Tunisia, (Social media playing big role in revolution across many countries. govt deliberately shut down)
% 
% - While we cannot keep them from filtering out service altogether,
% want to have private communication in plain sight (not stenographic)
% 
% - Strong crypto usable by people whispering to each other in streets
% 
% - Only trusted channel is not electronic (spoken word), to exchange key.
% 
% - Mention how one of our goals is to work within the 140 characters in that we want to fit our protocol as small as possible with as little encryption overhead.
% 
% - complimentary to Tor, solve problems Tor+Twitter does not
% 
% - What we are doing
% -- Define a protocol for users to communicate over an insecure public network like twitter with message confidentiality and subscriber anonymity. 
% 
% 
% 
% - Vocabulary
% }
