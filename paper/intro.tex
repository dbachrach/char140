\section{Introduction}

Recent events in Egypt, Tunisia, and many other countries have shown that social networking sites (Facebook, Twitter, and presumably others) played a non-trivial role in helping people organize themselves, plan protests, and distribute videos and other news to the outside world. Egypt was notable in that they eventually cut themselves off from the entire Internet, in a belated and ultimately ineffectual attempt to turn the tide. While it's difficult to draw overarching conclusions about the centrality of social networking versus more traditional means of communication in these important world events, it is clear that social media played a non-trivial role. Many other countries' leaders may well be worried of copycat revolutionaries. Other such countries may well try to censor or otherwise tamper with their citizens' use of social networking. To pick a current example: Syria appears to be attempting
a nationwide man-in-the-middle attack against Facebook~\cite{syria-facebook}. 

As a first step towards improving social network systems for such
environments, we seek to enable the use strong cryptographic primitives
overlaid on existing microblogging systems like Twitter, adding both
encryption and integrity to {\em tweets} (Twitter messages) among
groups. Keys must be shared to enable secure group communication, but we
should not rely on pre-arranged public key hierarchies or complex
protocols for key exchange. Users should be able to find tweets from
their group easily and then be able to receive those tweets with some
anonymity. Groups that may be targeted for their activities, even when
tweets are encrypted, should be offered {\em plausible deniability} that
they could be participating in another group instead.

%The goals of \hoot are to enable Twitter messages
%we propose \hoot, a system for censorship-resistant
%communication. The first step in designing such a system is
%As a first step toward improving social network systems for such environments, %we wish to design a system to enable the use of strong cryptographic primitives%, overlaid on existing microblogging services like Twitter. We wish to enable s%ecure group communication without requiring prearranged public key hierarchies %or other forms of key exchange. We wish to provide some measure of anonymity, b%y blurring the binding between the sender and recipients of any given message, %using existing innocuous messages as a form of cover traffic, providing some me%asure of deniability, at least for recipients of messages.

To achieve these goals, we propose \hoot, a system for
censorship-resistant microblogging. Our design features an interface in
which most users never see or concern themselves with cryptographic
keys. Instead, one of our key insights is that we can overload Twitter's
{\em hashtag} mechanism as a way of deriving cryptographic key
material. \hoot can be built on top of Twitter or another microblogging
system without modifying the underlying system. Through careful design,
encrypting and decrypting {\em \msgs} (\hoot messages) and bandwidth
overhead should be acceptably small at both the server and client sides.
\hoot makes it very difficult for the censor to distinguish between the
\msgs of a group with the \msgs of a select number of other groups.
%\hoot could have a remarkably small footprint in terms of additional
%computation and communication costs for both Twitter and its clients.

%Before we can sort out exactly how that should work, it's
%important to first understand how hashtags are used in the wild.

Hashtags are widely used in Twitter to label topics which others will
then subscribe to and follow. For example, most Usenix conferences adopt
the tag {\tt \#usenix}, allowing attendees to discuss the conference
with one another in real time. Political protests might end up using
several different tags (e.g., Egyptian discussions happen under {\tt
  \#tahrir}, {\tt \#jan25}, {\tt \#25jan}, and {\tt \#egypt}, among
others). Hashtags searches are generally case-insensitive.

Some tags have staggering volumes of messages. To pick a notable example, pop singer Justin Bieber asked his roughly 9 million followers to discuss his movie, {\em Never Say Never} using the hashtag {\tt \#nsn3d}. At its peak, roughly 1\% of Twitter's traffic mentioned this tag\footnote{Statistics via Trendistic, Topsy, and HashTracking.}. Since the movie's release in February 2011, there have been roughly 164 thousand tweets using {\tt \#nsn3d}, an average of 1.6 per minute with significantly higher peaks. 500 recent tweets on this hashtag generated 346 thousand impressions, reaching an audience of 212 thousand followers within a 24 hour period (measured in mid-April 2011). 

Of course, not all tags are as popular. We will show later, in Section~\ref{sec:experiments}, that hashtag usage follows a power-law distribution; a small number of hashtags are incredibly widely used and large numbers of tags are used very rarely or only once. We would like to design a system that can leverage these communications to create cover traffic for other, more sensitive messages, but without simply reusing the popular hashtag for other content. This will require converting hashtags into cryptographic keys and arranging for them to collide in some fashion, such that a query for {\tt \#nsn3d} and a query for a more sensitive tag are indistinguishable to an observer, thus providing some measure of deniability to group subscribers (``Protests? I'm just a fan of Justin Bieber!''). We also need to give some amount of control to the organizers of the sensitive communications, allowing them to select any popular hashtag with which they might prefer to collide.

Ultimately, we see two main paths to designing our system. One option would be to send encrypted messages that include the real {\tt \#nsn3d} tag, perhaps engineering some sort of steganographic process that tries to hide the plaintext within messages that are statistically similar to other posts from Bieber's fan, but it seems inappropriate to produce false messages like this. The other possibility is to imagine that {\em all} Twitter messages are encrypted in a uniform way, where knowing the plaintext of the hashtags would enable the decryption of a message. (It's easy to see a proxy server, of some sort, providing an ``encrypted'' interface to Twitter in this fashion% ; more on that in \hlfixme{TODO} Section~\ref{sec:whatever-else}
.) This is the design we chose to pursue. In this setup, we can encrypt and MAC every message with a random session key, which can be decrypted if the user knows the proper hashtag. ``Encrypted'' hashtags can also be generated by hashing the plaintext hashtags and truncating those hashes.  (We address this unwieldy vocabulary when we present our design in Section~\ref{sec:design}.) Consequently, two different plaintext hashtags can collide with each other with a probability related to the number of bits in the truncated hash.

\if 0
% decided not to use this, and instead convert "threat model" into "design goals"
To make a system like this ``real'', we must:
\begin{itemize}
\item Ensure that real Twitter messages use enough hashtags that, when reflected through our system, provide a significant amount of cover traffic in which to hide other messages.
\item Ensure that hashtags, when used directly as secrets, can be long enough to defeat computational brute force searches, yet be short enough to be memorized and passed along through spoken gossip.
\item Ensure that followers of secret hashtags have a defensible cover story (e.g., ``I'm just a big fan of Justin Bieber!'').
\item Ensure that those who post with secret hashtags can protect themselves from discovery.
\item Ensure that censorship systems, should they not know the secret hashtags, cannot distinguish those messages from other perfectly legitimate messages. We want to ensure that the only surefire way to filter secret messages is to disable the entire social network.
\item Be backward compatible, to the extent possible, with the real Twitter.
\item Minimize the extent to which we need to leverage external anonymity/censorship-resistance systems.
\end{itemize}
\fi

% Our work aims to offer modest improvements to the ability for groups of people to carry out conversations, via social media, 
% 
% \hl{
% - Problem: Twitter like semantics w/ encrypted messages
% 	- Follow a Hash tag
% 	- Take hash tag and create something with crypto strength
% 	- Something derived from tags you can search on
% 	- But also deliberate collisions (cover traffic)
% 	
% - Like to have thing that feels like twitter but anonymity properties:
% 
% - Twitter/Facebook relevant in Tunisia, (Social media playing big role in revolution across many countries. govt deliberately shut down)
% 
% - While we cannot keep them from filtering out service altogether,
% want to have private communication in plain sight (not stenographic)
% 
% - Strong crypto usable by people whispering to each other in streets
% 
% - Only trusted channel is not electronic (spoken word), to exchange key.
% 
% - Mention how one of our goals is to work within the 140 characters in that we want to fit our protocol as small as possible with as little encryption overhead.
% 
% - complimentary to Tor, solve problems Tor+Twitter does not
% 
% - What we are doing
% -- Define a protocol for users to communicate over an insecure public network like twitter with message confidentiality and subscriber anonymity. 
% 
% 
% 
% - Vocabulary
% }
