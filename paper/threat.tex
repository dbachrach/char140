\section{Threat model and system goals}

In this section, we briefly outline our threat model and then describe
the goals of our design.

\subsection{Threat model} \label{sec:threat}
\hoot is designed to provide censorship resistance against an adversary
who can observe all \hoot traffic and can block any tweets it chooses. In
practice, the adversary may only block tweets being received by a subset of users,
but this does not affect our model. The adversary seeks to identify and
block tweets discussing a small set of topics, based on the identity of
the sender, keywords in the content, or the hashtag itself. However,
our adversary does not want to disable the entire service, nor does
our adversary want to disable popular but innocuous discussion
threads; such crude censorship might
then generate additional unrest in the population. 
% The
% adversary is not willing to block the tweets on hashtags used for most
% communication, though it might be willing to block a small fraction of
% such tweets.
Furthermore, Kuppusamy and Shanmugam~\cite{ict-economy} showed that information technology 
and communication leads to greater economic growth. Broad censorship 
reduces the utility of the entire system, which is presumably used for
economic activities more valuable than discussing teenage pop stars.

%  Blocking a large number of tweets reduces the reliability
% and utility of the entire system, and {\em communication is great}.
% \hlfixme{fix this sentence. can we cite something showing the
%   relationship between communication and economic growth?} 

The adversary has moderate computing power: we assume that it can
perform a brute force key space search of up to 64 bits. The
Distributed.net distributed computing group cracked a 64-bit RC5 key in
2002 with computing power equivalent to approximately 46,000 2GHz
processors working for 790
days.\footnote{\url{http://www.distributed.net/images/9/92/20020925_-_PR_-_64_bit_solved.pdf}}
Extrapolating using Moore's Law, we assume that the adversary can
perform the same computation in approximately two weeks with today's
machines, and of course the adversary's computational strength will
only grow as time advances. We describe how
to address this in Section~\ref{sec:discuss}.

We assume that the \hoot server does not cooperate with the adversary. We
also assume that the adversary has no insiders in the group who leak the
group's secrets. Likewise, we assume the adversary has no avenues
to attack users, such as setting up a covert keylogger
on a group member's machine or coercing a group
member to divulge a secret. We concede that such attacks would effectively undermine \hoot's censorship
resistance. Note that this serves as a practical bound on any attack; if
it would be easier to establish an insider in the group or subvert one
of the client systems than to perform a computational attack, we will
consider our design to be successful.
We further discuss mitigations to such attacks in Section~\ref{sec:discuss}.

\subsection{System goals} \label{sec:goals}
Our goal is to allow a user to send a secure message to a private group
of individuals, allowing only the group members to read the plain-text
message, and to accomplish this with a user interface that looks and
feels much like the vanilla Twitter interface. Ultimately, this creates
a variety of constraints and challenges.

\begin{description}

\item[Simple key distribution.] To make the system as easy to use as
  possible, keys should be simple to create, distribute, and use. We
  therefore rule out any cryptographic key hierarchy such as a public
  key infrastructure or PGP/GPG key signing parties. Instead we wish
  to have keys that can literally be passed via
  word-of-mouth, from person to person in the group. We propose
  to derive keys from the group's plaintext hashtag, which effectively
  serves as the membership password. When a user subscribes to a given plaintext
  hashtag, she inputs the hashtag into her \hoot client, and the client
  derives the necessary keys.
%essentially use the plaintext hashtags as passwords for this
%  purpose. In HooT, a group's tweets are encrypted
%  using a symmetric key derived 
%We don't want to require any sort of prearranged cryptographic key
%hierarchy. We don't want any sort of complex key management process that
%would confuse unsophisticated users.  Instead, if the user knows the
%hashtag, that's all that's necessary to read or post messages.

\item[Confidentiality of tweets.] Since the plaintext hashtag is being
  used to generate encryption keys, it should have sufficient entropy to
  protect against dictionary attacks and brute force. This is in tension
  with our desire to have the plaintext hashtags be easily memorized and
  shared between users, ideally by voice alone.
%Plaintext hashtags within a message will
%  be converted into cryptographic keys to encrypt and MAC that message.
%  To keep an attacker from guessing the plaintext hashtags (with or
%  without brute-forcing searches), we must require these tags to have a
%  significant amount of entropy, yet this is in tension with our desire
%  to have the plaintext hashtags be something that users can share and
%  memorize by voice alone.

\item[Censorship resistance and denial of service.] While we do not attempt
  to defeat censorship of the \hoot service in its entirety, we seek to
  defeat attempts to censor specific groups and keywords. Perng et
  al.~\cite{perng05revisited} define {\em censorship susceptability} as the probability that the
  adversary can block a targeted message while allowing at least one
  other message to be received. This is a
  difficult requirement to meet in our system. We instead aim to allow
  only {\em heavy-handed censorship}, which we define as censorship of a
  group only through censorship of multiple, unrelated groups. By
  censoring these groups together, the adversary lowers the utility of
  the system as a whole. This is similar to the resistance provided by
  document-based systems like Tangler and Dagster~\cite{tangler,dagster}.
%a fraction of it. We want to ensure that no firewall, perhaps wanting to
%censor sensitive keywords, will be able to identify which messages are
%acceptable and which are forbidden. It's all or nothing.
% We also want to provide enough information that missing messages can
% be detected as being absent.

\item[Recipient anonymity.] To achieve censorship resistance, we rely on
  being able to protect recipient anonymity. Adapting the definition
  from Pfitzmann and Hansen~\cite{terminology}, we require that the
  recipients of the tweets, i.e. the group members, not be identifiable
  from among a larger set of possible recipients. \hoot makes this
  possible by mapping plaintext hashtags, which identify the groups, to
  short hashtags that can be made to collide with those of other
  groups. All of the recipients in all groups with colliding hashtags
  form a recipient anonymity set, with the tweets from other colliding
  groups providing cover traffic. \hoot can also be said to provide
  {\em subscriber anonymity}, as introduced by Mislove et al. in their
  description of AP3~\cite{ap3}. Hordes~\cite{hordes} and P5~\cite{P5}
  have similar requirements. The main additional feature of subscriber
  anonymity over recipient anonymity is that the act of subscribing
  should not reveal information that could be used to break recipient
  anonymity.

%That
%  plaintext hashtag needs to be mapped to something broader, which will
%  have a high likelihood of colliding with other unrelated hashtags. By
%  design, we want our system to use unrelated messages as {\em cover
%    traffic}.

\item[Recipient deniability.] If a \hoot user is under physical threat to
  reveal what hashtags she subscribes to, it's important that she can
  offer a convincing lie. Through careful selection of groups with
  colliding hashtags, she could name the hashtag of an innocuous group
  that could reasonably be of interest to members of the target group.
  Suitable choices can include trending topics (e.g., the Justin Bieber
  movie hashtag {\tt \#nsn3d}), socially-appropriate discussion groups
  (e.g. {\tt \#Bible} or {\tt \#Quran}), or topics that related to
  other innocuous professional or personal interests.
%  \hlfxnote{I specifically
%    avoided describing who might use these tags and leave it to the
%    reader to infer something like: young rebelious teens, paranoid tea
%    partiers, and hackers.}
%, such as naming a trendy yet innocuous hashtag
%  that happens to collide with the same set of messages that the
%  receiver is downloading.

\item[Sender anonymity or deniability.] Along those lines, we can only
  provide limited protection to a sender. A sender should gain some plausible
  deniability against a passive attacker, in that she could be
  tweeting about any possible topic that collides with her post.
%   Similarly, she nominally gains sender
%   anonymity in the sense that any member of any colliding group could
%   actually be sender.
 If, however, we must resist physical attacks against a sender,
coercing them to decrypt a posted message, our core \hoot design
will not protect them.
% mkw -- removed the backdoor idea for simplicity of the text.
%     -- if restored, I suggest citing a related backdoor idea and point
%     -- how expensive it would be.
% (e.g. a technique by which a message might have two valid
%  decryptions).
Instead, message senders who need to remain anonymous or who require the
ability to deny having posted a given message must use external means,
such as Tor, to connect to the \hoot service for posting messages. (If a
decentralized or P2P transport mechanism was used for microblogging,
like BirdFeeder~\cite{sandler09}, such a system could be extended to have
anonymous posting features. For this specific research, we are
generally targeting a centralized service more like Twitter.)

\item[Replay attacks.] It's possible that a malicious user, or even a
  malicious microblogging service, could not only remove messages but
  could also replay old messages, possibly with telling side effects
  (e.g. ``Meet in the town square at noon.''). We must have 
  mechanisms to reject duplicates.

\item[Statistical and traffic analysis.] Even if the adversary cannot
  decrypt messages, it may be able to learn things by scanning large
  populations of \hoot messages. While we make no explicit attempt to hide who the
  sender of a message might be (see ``sender anonymity,'' above), we do
  want to provide a strong degree of resistance to traffic analysis that
  might otherwise bind senders to receivers. Our system should make it
  difficult or impossible for observers to reconstruct the social graph.
% mkw -- just not gonna make this paper. Also may not be needed
%  \hlfixme{Should include discussion of social network privacy,
%    community privacy, adversarial community privacy, etc.}

\item[Secret informers and coerced users.] If a group member, whether
  sender or recipient, is an insider for the adversary or if the
  plaintext hashtag is stolen through a keylogger or coersion, the key
  is compromised and the group's messages will become readable and
  censorable to the adversary. While we cannot stop such attacks, we
  clearly need some form of key agility, to allow group organizers to
  distribute new hashtags to replace older, compromised hashtags.
%We assume that plaintext
%  hashtags can be shared by word of mouth, yielding something akin to a
%  cryptographic key distribution service. What if one of the recipients
%  turns to be a turncoat? What if a legitimate user has a keylogger or
%  otherwise-compromised computer? While we cannot defeat such turncoats,
%  we can offer some amount of key agility, where the sender can
%  distribute new hashtags to replace older, compromised hashtags.

\item[Compatibility.] We want to ensure that \hoot can be layered on top
  of Twitter, using existing Twitter mechanisms to search for and follow
  desired messages. We also must ensure that real Twitter users could
  incrementally migrate to using \hoot  as a service above the existing
  Twitter. To that end, we must demonstrate that we can implement
  efficient proxy servers, converting Twitter to \hoot to bootstrap an
  effective \hoot rollout.

\end{description}

%formalize the idea of which they 
One goal that we do not seek to achieve is {\em membership concealment},
which Vasserman et al. define as hiding the fact that the members are
participating in the system~\cite{vasserman09mcon}. The rationale for
membership concealment is that employing a tool designed to circumvent
the censor will draw unwanted attention to the user. By building \hoot
on top of a popular communications medium (Twitter), ideally with many
groups using \hoot in place of normal hashtags, we argue that \hoot
could be deployed in such a way that users are typically not aiming to
circumvent censorship. Since \hoot also provides message privacy,
authentication, integrity, and receiver anonymity, groups have other
reasons to use it instead of plaintext tweets besides censorship
resistance. If \hoot is widely adopted over Twitter for typical group
communication, a censor that blocks other systems for
censorship-resistance (such as Tor bridges), might not be willing to
block all \hoot messages. 

Whether this is the case in the real world is hard to discern. China
appears to have blocked iTunes for about 10 days in 2008 due to a
pro-Tibet album; however, it restored service while blocking the album
page itself~\cite{china-itunes}. Further, there was not a China-specific
iTunes service at the time. If the colliding groups are popular in the
country under censorship, then blocking results in the censorship being
widely seen inside the country and raises awareness of
censorship. Google discloses to users in China when their searches have
been modified\footnote{See
  \url{http://googleblog.blogspot.com/2006/02/testimony-internet-in-china.html}},
providing a similar type of awareness.


%According to Google Blogoscoped, while Google.cn censors its
%search results, it also reports to users that some search results are
%not showing~\cite{china-google}.

%, the suddenly blocking of popular Twitter groups
%cannot be done without being widely noticed inside the



%This is an advantage of \hoot that many
%systems for censorship-resistance do not provide.


%\section{Design Goals}

%\begin{description}
%\item[Simple key distribution.] We don't want to require any sort of prearrange%d cryptographic key hierarchy. We don't want any sort of complex key management% process that would confuse unsophisticated users.  Instead, if the user knows %the hashtag, that's all that's necessary to read or post messages.
%
%\item[Privacy.] Plaintext hashtags within a message will be used to encrypt and% verify those messages.  In order to keep an attacker from guessing the plainte%xt hashtags (with or without brute-forcing searches), we must require these tag%s to have a significant amount of entropy, yet this is in tension with our desi%re to have the plaintext hashtags be something that users can share and memoriz%e by voice alone.
%
%\item[Receiver anonymity.]  When a user subscribes to a given plaintext hashtag%, they will tell their client the hashtag in question. That plaintext hashtag n%eeds to be mapped to something broader, which will have a high likelihood of co%lliding with other unrelated hashtags. By design, we want our system to use unr%elated messages as {\em cover traffic}.
%
%\item[Receiver deniability.] One step further, if HooT users are under physical% threat to reveal what hashtags they subscribe to, it's important that they can% offer a convincing lie, such as naming a trendy yet innocuous hashtag that hap%pens to collide with the same set of messages that the receiver is downloading.% (``I'm just a big fan of Justin Bieber!'')
%
%\item[Censorship resistance and denial of service.] While we don't attempt to d%esign mechanisms to defeat censorship of the HooT service, in its entirety, we %want to defeat attempts to censor only a fraction of it. We want to ensure that% no firewall, perhaps wanting to censor sensitive keywords, will be able to ide%ntify which messages are acceptable and which are forbidden. It's all or nothin%g. % We also want to provide enough information that missing messages can be de%tected as being absent.
%
%\item[Sender anonymity or deniability.] Along those lines, we can only provide %limited protection to a sender. If a sender is physically threatened to decrypt% a posted message, we offer no attempt to build a backdoor into the system wher%e a message might have two valid decryptions. Instead, message senders who need% to remain anonymous or who require the ability to deny having posted a given m%essage must use external means, such as Tor, to connect to the HooT service for% posting messages. (If a decentralized or p2p transport mechanism was used for %microblogging, like BirdFeeder~\cite{sandler09}, the p2p transport could possib%ly be extended to have anonymous posting features.)
%
%\item[Replay attacks.] It's possible that a malicious user, or even a malicious% microblogging service, could not only remove messages but could also replay ol%d messages. We must have sufficient mechanisms to reject duplicates.
%
%\item[Statistical and traffic analysis.] Even if an observer cannot decrypt mes%sages, they may be able to learn things by scanning large populations of HooT m%essages. While we make no attempt to hide who the sender of a message might be %(see ``sender anonymity,'' above), we do want to provide a strong degree of res%istance to traffic analysis which might otherwise bind senders to receivers. Ou%r system should make it difficult or impossible for observers to reconstruct th%e social graph.
%
%\item[Secret informers and coerced users.] We are assuming that plaintext hasht%ags can be shared by word of mouth, yielding something akin to a cryptographic %key distribution service. What if one of the recipients turns to be a turncoat?% What if a legitimate user has a keylogger or otherwise-compromised computer? W%hile we cannot defeat such turncoats, we can offer some amount of key agility, %where the sender can distribute new hashtags to replace older, compromised hash%tags.
%
%\item[Compatibility.] We want to ensure that HooT can be layered on top of Twit%ter, using existing Twitter mechanisms to search for and follow desired message%s. We also must show that real Twitter users could incrementally migrate to usi%ng HooT above Twitter, and that HooT proxy servers could efficiently bootstrap %this process.
%
%\item 
%
%\end{description}
% 
%  We wish to guarantee as much privacy as possible via an open public timeline like that of Twitter. We now discuss the threat model involved in evaluating the design decisions for the security protocol.
% 
% There are two entities which can attack this system. One is the service provider for the communication like Twitter. The other is an active third party observer who is either trying to gather information about what is being said or to whom it is being said.
%  
% Imagine an evil Twitter that can maliciously tamper with any of the tweets posted. Since Twitter has full control of the service, they act as an empowered man-in-the-middle between a secure message sender and the recipient group members. Our goals are to prevent Twitter or a third party from reading, creating, or altering a secure message, or discovering who the group recipients are.
% 
% Clearly, we cannot directly protect the identity of the secure message sender. A message must be posted to Twitter from some account. By definition, Twitter must know who that user is. We do not consider this limitation a security flaw, since it is fundamental to the problem description. However, if sender anonymity is required, tools like Tor could provide the needed indirection.
% 
% Twitter can simply refuse to post a tweet, which is a simple Denial of Service attack. Like any platform, protecting against DOS is nearly impossible. A slight modification to this, is the DDOS, which can be performed by other malicious Twitter users. A malicious user can deliberate tweet hundreds of thousands of messages with the public identifier of a group, which could possibly overload a user's search for the identifier with noise. Later we will describe how we prevent against this action and in fact utilize it to provide added security.
% 
% An attack that can be administered by Twitter or a third party is a replay attack. Since Twitter is a public forum, anyone can see the encrypted texts posted, and nothing prevents someone from simply copying such a message and posting it again. We want to protect against this threat.
% 
% A malicious Twitter also can take a valid Hoot and change the associated author or time that it displays with it. Our system does not directly prevent against these attacks, but they can be addressed by including the time stamp and author name with in the plain text of the message.
% 
% We want to prevent Twitter or any third party from gathering trending data on the encrypted posts. Even if they cannot read the messages, they can observe that someone is posting to the same group of recipients following certain patterns (such as daily at X time, or after major events), this repetition could compromise the group's identity by essentially creating a profile. This type of information is less of a concern to users of Twitter that are simply chatting, but one can imagine that this information can end up being used to identify spies. For example, by observing a pattern between company secrets being leaked to some competitor and a particular employee's secret tweets to a group just hours prior to each incident, the leaker could be identified. This attack can be prevented by using anonymizing services like Tor coupled with a collision technique we will describe later to create ``cover traffic.''
% 
% Finally, there are certain aspects of security which we placing beyond the scope of this protocol. Key distribution, is something we do not address. Ours is a world where the key must be distributed through outside channels, either exchanged over a known secure channel, or whispered through the streets of Cairo. We especially want our protocol to take advantage of such a `whisper channel', and be easy enough to use and reconfigure that a lay person can use it with little learning curve. \hl{insert Why Johnny can't encrypt discussion and reference here maybe?} Protecting against double agents is also a real concern in certain circles, and while we make no effort to protect against this, the ease of distribution of the key should allow a group to expunge members and regroup with ease. Our goal is to make the Crypto portions of the protocol sufficiently robust that an attacker would find it easier to use a double agent than to crack the protocol. At this point, our work is essentially done, and member management is left to group itself.
