\section{Threat Model}

Our goal is to allow a user to send a secure message to a private group of individuals allowing only the group members to read the plain-text message, and to accomplish this with a user interface that looks and feels much like the vanilla Twitter interface. Ultimately, this creates a variety of constrains and challenges.

\begin{description}
\item[Privacy] Plaintext hashtags within a message will be converted into cryptographic keys to encrypt and MAC that message.  In order to keep an attacker from guessing the plaintext hashtags (with or without brute-forcing searches), we must require these tags to have a significant amount of entropy, yet this is in tension with our desire to have the plaintext hashtags be something that users can share and memorize by voice alone.

\item[Receiver anonymity]  When a user subscribes to a given plaintext hashtag, they will tell their client the hashtag in question. That plaintext hashtag needs to be mapped to something broader, which will have a high likelihood of colliding with other unrelated hashtags. By design, we want our system to use unrelated messages as {\em cover traffic}.

\item[Receiver deniability] One step further, if HooT users are under physical threat to reveal what hashtags they subscribe to, it's important that they can offer a convincing lie, such as naming a trendy yet innocuous hashtag that happens to collide with the same set of messages that the receiver is downloading.

\item[Censorship resistance and denial of service] While we don't attempt to design mechanisms to defeat censorship of the HooT service, in its entirety, we want to defeat attempts to censor only a fraction of it. We want to ensure that no firewall, perhaps wanting to censor sensitive keywords, will be able to identify which messages are acceptable and which are forbidden. It's all or nothing. We also want to provide enough information that missing messages can be detected as being absent.

\item[Sender anonymity or deniability] Along those lines, we can only provide limited protection to a sender. If a sender is physically threatened to decrypt a posted message, we offer no attempt to build a backdoor into the system where a message might have two valid decryptions. Instead, message senders who need to remain anonymous or who require the ability to deny having posted a given message must use external means, such as Tor, to connect to the HooT service for posting messages. (If a decentralized or p2p transport mechanism was used for microblogging, like BirdFeeder~\cite{sandler09}, this might offer an alternative method of posting anonymously.)

\item[Replay attacks] It's possible that a malicious user, or even a malicious microblogging service, could not only remove messages but could also replay old messages. We must have sufficient mechanisms to reject duplicates.

\item[Statistical and traffic analysis] Even if an observer cannot decrypt messages, they may be able to learn things by scanning large populations of HooT messages. While we make no attempt to hide who the sender of a message might be (see ``sender anonymity,'' above), we do want to provide a strong degree of resistance to traffic analysis which might otherwise bind senders to receivers. Our system should make it difficult or impossible for observers to reconstruct the social graph.

\item[Secret informers and coerced users] We are assuming that plaintext hashtags can be shared by word of mouth, yielding something akin to a cryptographic key distribution service. What if one of the recipients turns to be a turncoat? What if a legitimate user has a keylogger or otherwise-compromised computer? While we cannot defeat such turncoats, we can offer some amount of key agility, where the sender can distribute new hashtags to replace older, compromised hashtags.

\end{description}
% 
%  We wish to guarantee as much privacy as possible via an open public timeline like that of Twitter. We now discuss the threat model involved in evaluating the design decisions for the security protocol.
% 
% There are two entities which can attack this system. One is the service provider for the communication like Twitter. The other is an active third party observer who is either trying to gather information about what is being said or to whom it is being said.
%  
% Imagine an evil Twitter that can maliciously tamper with any of the tweets posted. Since Twitter has full control of the service, they act as an empowered man-in-the-middle between a secure message sender and the recipient group members. Our goals are to prevent Twitter or a third party from reading, creating, or altering a secure message, or discovering who the group recipients are.
% 
% Clearly, we cannot directly protect the identity of the secure message sender. A message must be posted to Twitter from some account. By definition, Twitter must know who that user is. We do not consider this limitation a security flaw, since it is fundamental to the problem description. However, if sender anonymity is required, tools like Tor could provide the needed indirection.
% 
% Twitter can simply refuse to post a tweet, which is a simple Denial of Service attack. Like any platform, protecting against DOS is nearly impossible. A slight modification to this, is the DDOS, which can be performed by other malicious Twitter users. A malicious user can deliberate tweet hundreds of thousands of messages with the public identifier of a group, which could possibly overload a user's search for the identifier with noise. Later we will describe how we prevent against this action and in fact utilize it to provide added security.
% 
% An attack that can be administered by Twitter or a third party is a replay attack. Since Twitter is a public forum, anyone can see the encrypted texts posted, and nothing prevents someone from simply copying such a message and posting it again. We want to protect against this threat.
% 
% A malicious Twitter also can take a valid Hoot and change the associated author or time that it displays with it. Our system does not directly prevent against these attacks, but they can be addressed by including the time stamp and author name with in the plain text of the message.
% 
% We want to prevent Twitter or any third party from gathering trending data on the encrypted posts. Even if they cannot read the messages, they can observe that someone is posting to the same group of recipients following certain patterns (such as daily at X time, or after major events), this repetition could compromise the group's identity by essentially creating a profile. This type of information is less of a concern to users of Twitter that are simply chatting, but one can imagine that this information can end up being used to identify spies. For example, by observing a pattern between company secrets being leaked to some competitor and a particular employee's secret tweets to a group just hours prior to each incident, the leaker could be identified. This attack can be prevented by using anonymizing services like Tor coupled with a collision technique we will describe later to create ``cover traffic.''
% 
% Finally, there are certain aspects of security which we placing beyond the scope of this protocol. Key distribution, is something we do not address. Ours is a world where the key must be distributed through outside channels, either exchanged over a known secure channel, or whispered through the streets of Cairo. We especially want our protocol to take advantage of such a `whisper channel', and be easy enough to use and reconfigure that a lay person can use it with little learning curve. \hl{insert Why Johnny can't encrypt discussion and reference here maybe?} Protecting against double agents is also a real concern in certain circles, and while we make no effort to protect against this, the ease of distribution of the key should allow a group to expunge members and regroup with ease. Our goal is to make the Crypto portions of the protocol sufficiently robust that an attacker would find it easier to use a double agent than to crack the protocol. At this point, our work is essentially done, and member management is left to group itself.
